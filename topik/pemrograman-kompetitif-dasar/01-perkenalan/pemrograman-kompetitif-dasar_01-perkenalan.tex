\input{../config.tex}

\title{Perkenalan}
\author{Tim Olimpiade Komputer Indonesia}
\date{}

\begin{document}

\begin{frame}
\titlepage
\end{frame}

\begin{frame}
\frametitle{Apa itu Pemrograman Kompetitif?}
\begin{block}{Pemrograman Kompetitif}
\foreignTerm{Competitive programming is a mind sport usually held over the Internet or a local network, involving participants trying to program according to provided specifications.}

-Wikipedia-
\end{block}
\end{frame}

\begin{frame}
\frametitle{Ajang Pemrograman Kompetitif}
\begin{itemize}
  \item International:
    \begin{itemize}
    	\item IOI
    \end{itemize}
  \item Nasional:
    \begin{itemize}
    	\item OSN: Kemendikbud
    	\item Compfest: UI
    	\item NPC: ITS
    	\item ILPC: Ubaya
    \end{itemize} 
\end{itemize}

\end{frame}

\begin{frame}
\frametitle{Apa yang Dilakukan di Pemrograman Kompetitif?}
Yang dilakukan adalah melakukan \foreignTerm{problem solving} (bukan membuat web atau yang lain).
\\
\foreignTerm{Problem solving} di pemrograman kompetitif berupa menyelesaikan sebuah/beberapa persoalan yang telah ada solusinya.
\end{frame}

\begin{frame}
\frametitle{Demonstrasi Problem Solving}
\begin{block}{Masalah}
Terdapat 3 jenis koin, dengan nilai 1, 5 dan 10.
berapakah jumlah koin minimal yang dibutuhkan untuk mendapatkan nilai N ?
\end{block}
\end{frame}

\begin{frame}
\frametitle{Demonstrasi Problem Solving (lanj.)}
\begin{itemize}
  \item Masalahnya : memecah bilangan N menjadi bentuk $a + 5 * b + 10 * c$ dan $a + b + c$ minimal
  \item Data : Koin yang tersedia bernilai 1, 5 dan 10
  \item Observasi : Mengambil 1 koin bernilai 5 akan lebih baik dibandingkan mengambil 5 koin bernilai 1. Begitu pula dengan koin 5 dan 10
\end{itemize}
\end{frame}

\begin{frame}
\frametitle{Demonstrasi Problem Solving (lanj.)}
\begin{block}{Kesimpulan}
Untuk mendapatkan hasil terbaik, kita mecoba untuk mengambil koin 10 sebanyak banyaknya, kemudian koin 5 dan kemudian koin 1
\end{block}
\end{frame}

\end{document}
