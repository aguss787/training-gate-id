\input{../config.tex}

\title{Perkenalan}
\author{Tim Olimpiade Komputer Indonesia}
\date{}

\begin{document}

\begin{frame}
\titlepage
\end{frame}

\begin{frame}
\frametitle{Apa itu Pemrograman Kompetitif?}
\begin{block}{Pemrograman Kompetitif}
\foreignTerm{Competitive programming is solving well-defined problems by writing computer programs under specified limits.}

-Ashar Fuadi-
\end{block}
\end{frame}

\begin{frame}
\frametitle{Ajang Pemrograman Kompetitif}
\begin{itemize}
  \item International:
    \begin{itemize}
    	\item IOI
    \end{itemize}
  \item Nasional:
    \begin{itemize}
    	\item OSN: Kemendikbud
    	\item Compfest: UI
    	\item NPC: ITS
    	\item ILPC: Ubaya
    \end{itemize} 
  \item Online:
    \begin{itemize}
      \item Codeforces
      \item Codechef
      \item Topcoder SRM
    \end{itemize}
\end{itemize}

\end{frame}

\begin{frame}
\frametitle{Apa yang Dilakukan di Pemrograman Kompetitif?}
\begin{itemize}
  \item Yang dilakukan adalah menyelesaikan satu atau beberapa \foreignTerm{well-defined problems} dengan membuat program dalam batas-batas yang ditentukan
  \item Yang dimaksud \foreignTerm{well-defined problem} adalah sebuah masalah yang telah terdefinsi dengan baik (asumsi yang diperlukan dan lain-lain)
  \item Batas yang ditentukan: batas waktu, memory dan lain-lain
\end{itemize}
\end{frame}

\begin{frame}
\frametitle{Demonstrasi Problem Solving}
\begin{block}{Masalah}
Terdapat 3 jenis koin, dengan nilai 1, 5 dan 10.
berapakah jumlah koin minimal yang dibutuhkan untuk mendapatkan nilai N ?
\end{block}
\end{frame}

\begin{frame}
\frametitle{Demonstrasi Problem Solving (lanj.)}
\begin{itemize}
  \item Masalahnya : memecah bilangan N menjadi bentuk $a + 5 * b + 10 * c$ dan $a + b + c$ minimal
  \item Data : Koin yang tersedia bernilai 1, 5 dan 10
  \item Observasi : Mengambil 1 koin bernilai 5 akan lebih baik dibandingkan mengambil 5 koin bernilai 1. Begitu pula dengan koin 5 dan 10
\end{itemize}
\end{frame}

\begin{frame}
\frametitle{Demonstrasi Problem Solving (lanj.)}
\begin{block}{Kesimpulan}
Untuk mendapatkan hasil terbaik, kita mecoba untuk mengambil koin 10 sebanyak banyaknya, kemudian koin 5 dan kemudian koin 1
\end{block}
\end{frame}

\begin{frame}
\frametitle{Bacaan lanjut}
  Penjelasan Ashar Fuadi mengenai pemrograman kompetitif: \textcolor{blue}{\href{https://www.quora.com/What-is-competitive-programming-2/answer/Ashar-Fuadi}{Quora}}
\end{frame}

\end{document}
