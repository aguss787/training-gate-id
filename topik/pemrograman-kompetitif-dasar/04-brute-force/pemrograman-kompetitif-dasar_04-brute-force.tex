\input{../config.tex}

\title{Brute Force}
\author{Tim Olimpiade Komputer Indonesia}
\date{}

\begin{document}

\begin{frame}
\titlepage
\end{frame}

\begin{frame}
\frametitle{Pendahuluan}
Melalui dokumen ini, kalian akan:
\begin{itemize}
  \item Mempelajari konsep pencarian \foreignTerm{brute force}
  \item Mampu mengerjakan persoalan dengan pendekatan \foreignTerm{brute force}
\end{itemize}

\end{frame}

\begin{frame}
\frametitle{Soal: Subset Sum}
Deskripsi:
\begin{itemize}
  \item Diberikan $N$ buah integer $(a_1, a_2, ..., a_N)$ dan integer $K$
  \item Apakah terdapat subset dari integer-integer tersebut sehingga jumlahan dari elemen subset tersebut sama dengan $K$? 
  \item Bila iya, maka keluarkan "YA". Selain itu keluarkan "TIDAK"
\end{itemize}

Batasan:
\begin{itemize} 
  \item $1 \leq N \leq 15$
  \item $1 \leq K \leq 10^9$
  \item $1 \leq a_i \leq 10^9$
\end{itemize}

\end{frame}

\begin{frame}
\frametitle{Solusi}
\begin{itemize}
  \item Untuk setiap elemen, kita memiliki 2 pilihan yaitu memilih elemen tersebut atau tidak memilihnya.
  \item Jika jumlahan dari elemen-elemen yang dipilih sama dengan $K$, maka terdapat solusi.
\end{itemize}
\end{frame}

\begin{frame}[fragile]
\frametitle{Solusi (Lanj.)}
\begin{lstlisting}
solve(pos, sum):
  if pos > n then
    if sum == K then
      return true
    else
      return false

  return solve(pos + 1, sum) 
      or solve(pos + 1, sum + a[pos])

if solve(1,0) == true then
  print "YA"
else
  print "TIDAK"
\end{lstlisting}
\end{frame}

\begin{frame}
\frametitle{Optimisasi}
Bisakah solusi tersebut menjadi lebih cepat?
gunakan \newTerm{pruning}!

\begin{block}{Pruning}
  Merupakan optimisasi dengan mengurangi \foreignTerm{search space} dengan cara menghindari pencarian pada jawaban yang pasti salah.
\end{block}
\end{frame}

\begin{frame}[fragile]
\frametitle{Solusi Optimasi}
\begin{lstlisting}
solve(pos, sum):
  if sum > K then
    return false

  if pos > n then
    if sum == K then
      return true
    else
      return false

  return solve(pos + 1, sum) 
      or solve(pos + 1, sum + a[pos])

if solve(1,0) == true then
  print "YA"
else
  print "TIDAK"
\end{lstlisting}
\end{frame}

\begin{frame}
\frametitle{Penjelasan}
\begin{itemize}
  \item Karena semua $a_i$ bernilai positif, maka \id{sum} tidak akan menjadi lebih kecil.
  \item Karena itu, bila \id{sum} sudah lebih besar dari \id{K} bisa dipastikan tidak akan tercapai sebuah solusi
\end{itemize}
\end{frame}

\begin{frame}
\frametitle{TODO Konten}
\begin{itemize}
  \item Berikan contoh lagi yang lebih sulit, dan tunjukkan bahwa kadang2 kita harus pintar dalam menentukan apa yang mau di-brute force
  \item Ajarkan untuk hitung kompleksitasnya. Kalau cukup untuk AC, langsung coding aja
\end{itemize}
\end{frame}

\end{document}
